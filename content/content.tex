% 3/5 21.43
\chapter{Introducción}

\begin{enumerate}
\item  Item1
\item  Item2
\item  Item3
\item  Item4
\end{enumerate}



\chapter{Estado del arte}
\section{Herramientas de...} 


\subsection{Herramienta 1}

\subsection{Conclusiones}

\subsubsection{Ejemplo de gráfica y tabla}

\begin{figure}[H]
\centering
\includegraphics*[width=\textwidth]{figures/SCORM_Data_Model_API.png} 
\caption{Uso del modelo de datos de SCORM con la API \cite{RTE2004}}
\end{figure}
En la tabla siguiente se detalla el modelo de datos de SCORM 2004 y el significado de cada campo. Para versiones anteriores varía ligeramente, pero en esencia el modelo continúa siendo el mismo:
\begin{table}[H]
\begin{tabular}{|p{2.1in}|p{3.8in}|} \hline 
\textbf{Elemento} & \textbf{Descripción} \\ \hline 
\textbf{cmi.\_version} & Representa la versión del modelo de datos \\ \hline 
\textbf{cmi.comments\_from\_learner } & Listado de comentarios del alumno \\ \hline 
\textbf{cmi.comments\_from\_lms} & Comentarios procedentes del LMS \\ \hline 
\textbf{cmi.completion\_status} & Indica si el alumno ha completado el SCO \\ \hline 
\textbf{cmi.completion\_threshold} & Determina a partir de qué porcentaje se considera que un SCO se ha completado \\ \hline 
\textbf{cmi.credit} & Indica si al alumno se le va a asignar una calificación a partir de su interacción con el SCO \\ \hline 
\textbf{cmi.entry} & Determina si es la primera vez que el alumno accede al SCO \\ \hline 
\textbf{cmi.exit} & Indica el motivo por el cual el alumno abandonó el SCO \\ \hline 
\textbf{cmi.interactions} & Define la información pertinente a la interacción del alumno con cada actividad evaluable con el objetivo de evaluar su actuación \\ \hline 
\textbf{cmi.launch\_data} & Datos provistos al SCO al ejecutar el curso, provenientes del manifIesto presente en el LMS \\ \hline 
\textbf{cmi.learner\_id} & Identificador del alumno en el LMS \\ \hline 
\textbf{cmi.learner\_name} & Nombre del alumno en el LMS \\ \hline 
\textbf{cmi.learner\_preference} & Especifica las preferencias del alumno en cuanto al SCO \\ \hline 
\textbf{cmi.location} & Especifica la posición actual del alumno dentro del SCO \\ \hline 
\textbf{cmi.max\_time\_allowed} & Máximo tiempo que se permite consumir un SCO \\ \hline 
\textbf{cmi.mode} & Identifica en qué modo se está consumiendo el curso (normal, revisión o previsualización) \\ \hline 
\textbf{cmi.objectives} & Especifica objetivos de evaluación para un SCO \\ \hline 
\textbf{cmi.progress\_measure} & Mide el progreso que ha realizado el alumno de cara a completar el SCO \\ \hline 
\textbf{cmi.scaled\_passing\_score} & Puntuación que se requiere para aprobar el SCO (en formato decimal 0-1) \\ \hline 
\textbf{cmi.score} & Especifica la puntuación del SCO para ese estudiante \\ \hline 
\textbf{cmi.session\_time} & Cantidad de tiempo que el estudiante ha pasado en el SCO en la sesión actual \\ \hline 
\textbf{cmi.success\_status} & Indica si el alumno ha aprobado el SCO \\ \hline 
\textbf{cmi.suspend\_data} & Provee espacio para persistir información entre sesiones \\ \hline 
\end{tabular}
\caption[Modelo de datos de SCORM]{Modelo de datos de SCORM}
\label{table:name}
\end{table}
